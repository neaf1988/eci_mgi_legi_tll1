%++++++++++++++++++++++++++++++++++++++++
% Don't modify this section unless you know what you're doing!
\documentclass[
letterpaper,
12pt,
singlespacing,
headsepline]{article}

\usepackage{tabularx} % extra features for tabular environment
\usepackage{amsmath}  % improve math presentation
\usepackage{graphicx} % takes care of graphic including machinery
\usepackage[margin=1in,letterpaper]{geometry} % decreases margins
\usepackage[final]{hyperref} % adds hyper links inside the generated pdf file
\usepackage[utf8]{inputenc} % Required for inputting international characters
\usepackage[T1]{fontenc} % Output font encoding for international characters
\usepackage[spanish]{babel}
\usepackage{mathpazo} % Use the Palatino font by default



\usepackage[backend=bibtex,style=authoryear,natbib=true]{biblatex} % Use the bibtex backend with the authoryear citation style (which resembles APA)

\addbibresource{references/references1.bib} % The filename of the bibliography

\usepackage[autostyle=true]{csquotes} % Required to generate language-dependent quotes in the bibliography
\hypersetup{
	colorlinks=true,       % false: boxed links; true: colored links
	linkcolor=blue,        % color of internal links
	citecolor=blue,        % color of links to bibliography
	filecolor=magenta,     % color of file links
	urlcolor=blue         
}

\newcommand{\grad}{$^{\circ}$}
%++++++++++++++++++++++++++++++++++++++++


\begin{document}
	\begin{titlepage}
	\vspace*{\stretch{0.5}}
	   \begin{center}
			\includegraphics[scale=0.5]{Images/1110_logotipo_institucional_vm300px.png} \\
		    \Large{\textbf{Escuela Colombiana de Ingeniería Julio Garavito}\vspace{2cm}}\\
		    \textbf{Maestría en Gestión de Información}\vspace{1cm}\\
		    \textbf{Ley en Gestión de Información}\vspace{1cm}\\
			\textsc{Taller 1} \vspace{2cm}  \\   
			\emph{Autor:} \textit{Ing. Fabio Quintero}\vspace{5cm}\\
	     \textit{\today}\\

	   \end{center}
	   \vspace*{\stretch{2.0}}
	\end{titlepage}
	






\section{Derechos de Autor}

\subsection{Participación ciudadana}
Acudiremos a una definicion para sustentar que la participación de este ciudadano suponiendo que la idea propuesta de esta persona estuvo debidamente documentada por su autor y que en la página se ejecuto de forma literal, lo cual creo estará sujeto a demostrar,  generará unos derechos morales por su obra pero que no cuenta con derechos patrimoniales sobre la misma, la definición de obra colectiva \footnote{Ley 23 de 1982 Articulo 8} en la cual un grupo de autores participan en la obra pero que es coordinada, divulgada y publicada por una persona natural y jurídica, como en este caso un grupo de ciudadanos coordinados por la UGPP y la DIAN participan en un proyecto plasmando sus ideas en el mismo. 

\subsection{Participación como opción de grado}
Todo estudiante de la Escuela Colombiana de Ingeniería por reglamento debe firmar para sus documentos de grado una cesión de derechos patrimoniales, pero que en todo caso puede ser negociable con el estudiante o terceros, pero suponiendo que el estudiante firma sin algún acuerdo diferente la ECI tendrá los derechos patrimoniales de tal trabajo pero el  estudiante tiene todos los derechos morales sobre la misma y debera reconocerle el mismo como se indica en la ley sobre derechos de autor en su artículo 30.
\\
La UGPP y la DIAN si no acordaron nada diferente en la cesión de derechos del proyecto del estudiante y este acompañamiento no fue sentado mediante un contrato por la prestación del servicio, deberán reconocer a la ECI los derechos patrimoniales de la plataforma, pero si por el contrario se definió un contrato se entenderá que la obra sera definida por encargo \footnote{Ley 23 de 1982 Articulo 20} y no deberá reconocer algún derecho a terceros.


\section{Protección de datos}
\subsection{Acceso a la información pública}
Acorde a la ley de acceso de transferencia y del derecho a acceso a la información pública \footnote{Ley 1712 de 2014} la UGPP y la DIAN están obligados, como se expresa en el Artiículo 5 literal \textbf{\textit{a}}, toda entidad pública a todos los niveles, siendo así se podría solicitar toda la información requerida sin perjucio y/o explicación alguna por parte del requieriente, pero teniendo en cuenta el caso planteado y como se explica en la subsiguiente sección ~\ref{subsec:semiprivados} los datos que la DIAN en calidad de Operador de la información \footcite{Ley 1266 de 2008 Articulo 3, Definiciones}, obtuvo esta información pro de cumplir su misión insitucional, es decir garantizar que todos los habitantes del territorio Colombiano cumplan con sus obligaciones tributarias, por lo tanto la información generada por la UGPP de un titular es de interés de la DIAN para el cumplimiento tributario del mismo a lo cual implica que esta información privada se convierte en semiprivada por definición.\footcite{Ley 1266 de 2008 Articulo 5, Definiciones - Dato Semiprivado}.
Entonces si los datos en específico que se transfirieron entre estas 2 entidas son semiprovados, deberemos acudir LEY 1712 Artículo 6 literal \textit{c}, \textsc{Información pública clasificada}, donde se espcifica que esta información será pública clasificada por la naturaleza previamente descrita.
\\
Podemos concluir que cualquier ciudadano podrá solicitar la información que tenga a bien y que la DIAN y la UGPP con administradores deberán garantizar la privacidad de los titulares de la información y deberan entregar dicha información clasificada sin perjucio de la intimidad de los titulares.


\subsection{Datos semiprivados}
\label{subsec:semiprivados}
Teniendo en cuenta el Artículo 13 de la ley de protección de datos personales\footnote{Ley Estatutaria 1581 de 2012}, estas 2 entidades públicas están autorizadas a obtener y consultar la información de las personas que lo entiendan como necesario, por lo tanto no requieren hacer ninguna clase de solicitud, como pedir la autorización del titular de los datos.
\\
Si bien cualquier ciudadano tiene el derecho constitucional de que se garantice su intimidad ''Todas las personas tienen derecho a su intimidad personal y familiar y a su  buen nombre, y el Estado debe respetarlos y hacerlos respetar \dots''\footnote{Constitución Política de Colombia, Articulo 15}, la ley de hábeas data\footnote{Ley Estatutaria 1266 de 2008} es garante de regular la información crediticia, comercial y servicios entre otros, entonces el titular de la información acudirá a esta pensando que puede bloquear el intercambio de esta información entre la UGPP y la DIAN, es de aclarar que por normatividad la UGPP debe contar con la autorización del titular para el uso de datos privados, por lo tanto el titular podría pensar que esta autorización es solo para la UGPP y no la DIAN y que estas 2 entidades tiene misiones diferentes y no deberían poder usar los datos sin su autorización, pero la misionalidad de este intercambio de datos es garantizar que cualquier persona con las normas relacionadas al pago de impuestos y no los evada, por lo tanto esta ley no impide que estos datos sean intercambiados.
\\
Se define un dato  semi privado a seguir ''Es semiprivado el dato que no tiene naturaleza íntima, reservada, ni pública y cuyo conocimiento o divulgación puede interesar no sólo a su titular sino a cierto sector o grupo de personas o a la sociedad en general, como el dato financiero y crediticio de actividad comercial o de servicios a que se refiere el Título IV de la presente ley''. Esta definición y lo anterior para hacer cumplir la función de la DIAN y que no exista alguien (incluyendo al titular) que evada impuestos los datos se convertirán en semiprivados.



%----------------------------------------------------------------------------------------
%	BIBLIOGRAPHY
%----------------------------------------------------------------------------------------

\printbibliography[heading=bibintoc]


\end{document}
